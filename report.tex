\documentclass[10pt,twocolumn]{article}

% Följande rad ska göra det möjligt att använda svenska bokstäver, som å, ä, ö. Kravet är 
% då att filen sparas i UTF-8-format. Om detta inte fungerar för dig, så kan du alltid 
% använda dig av {\aa} för å, \"a för ä och \"o för ö.
\usepackage[utf8]{inputenc}

\usepackage[swedish]{babel}

% Följande väljer typsnitt som är kloner av Times New Roman, Helvetica och lämpliga till
% dem anpassade matematiktypsnitt.
\usepackage{newtxtext}
\usepackage{newtxmath}

%  Programkod
\usepackage{listings}
\usepackage{color} %red, green, blue, yellow, cyan, magenta, black, white
\definecolor{mygreen}{RGB}{28,172,0} % color values Red, Green, Blue
\definecolor{mylilas}{RGB}{170,55,241}
\definecolor{gray}{RGB}{88,88,88}
\lstset{language=Matlab,%
    %basicstyle=\color{red},
    breaklines=true,%
    morekeywords={matlab2tikz},
    keywordstyle=\color{blue},%
    morekeywords=[2]{1}, keywordstyle=[2]{\color{black}},
    identifierstyle=\color{black},%
    stringstyle=\color{mylilas},
    commentstyle=\color{mygreen},%
    showstringspaces=false,%without this there will be a symbol in the places where there is a space
    numbers=left,%
    numberstyle={\tiny \color{gray}},% size of the numbers
    numbersep=5pt, % this defines how far the numbers are from the text
    emph=[1]{for,end,break},emphstyle=[1]\color{red}, %some words to emphasise
    %emph=[2]{word1,word2}, emphstyle=[2]{style},    
}

% Figurer
\usepackage{graphicx}
\graphicspath{{img/}}
\usepackage{float}

% Citat
\usepackage{dirtytalk}

\raggedbottom
\sloppy

\title{Laborationsrapport i TSKS10 \emph{Signaler, Information och Kommunikation}}

\author{Matti Lundgren \\ matlu703, xxyyzz-ååää }

\date{\today}

\begin{document}

\maketitle

\clearpage

\section{Inledning}

Laborationen går ut på att demodulera en smalbandig I/Q-modulerad signal där respektive I/Q-komponent bär en melodi följt av ett talspråk i form av hörbart ljud. 

Det är givet att den utsända signalen har utseendet $x(t)=x_I(t)\cos(2\pi f_ct)-x_Q(t)\sin(2\pi f_ct)+w(t)+z(t)$ där $w(t) = 0{,}001(\cos(2\pi f_1t) + \cos(2\pi f_2t))$ och $z(t)$ är en summa av signaler som sänds på andra bärfrekvenser. På grund av ekoeffekter blir den mottagna signalen $y(t)=x(t-\tau_1)+0{,}9x(t-\tau_2)$.

Annan relevant information som är given av laborationshandledningen listas nedan:
\begin{itemize}
\item Bärfrekvensen $f_c$ är någon av följande frekvenser: 18, 37, 56, 75, 94, 113, 132, 151 kHz.
\item $f_1$ och $f_2$ är multiplar av 1 Hz och överlappar inte med någon av de burna signalerna.
\item $\tau_2>\tau_1$, $\tau_2-\tau_1<500$ ms och $\tau_2-\tau_1$ är en multipel av 1 ms.
\end{itemize}

\section{Metod}



För att genomföra laborationen används, utöver laborationshandledningen, kursboken \textit{Signals, Informations and Communications} (utkast 14273) och Matlab. \linebreak Programkoden bifogas på sida 4 och de plottade figurerna på sida 3.

$y(t)$s amplitudspektrum plottas i figur 1. Ur spektrumet identifieras tre möjliga bärfrekvenser; 56, 94 och 151 kHz. De burna signalerna benämns hädanefter som $y_1(t)$, $y_2(t)$ respektive $y_3(t)$. Signalerna erhålls genom att bandpassfiltrera $y(t)$ runt respektive bärfrekvens. Då den sökta signalen består av hörbart ljud sätts den intressanta bandbredden till $B=20$ kHz. Signalerna plottas i figur 2. Av figuren att döma ser $y_1(t)$ ut att vara den sökta nyttosignalen i och med att den består av två tydliga delar som skiljs åt av en paus. $y_2(t)$ ser ut som brus och den senare delen av $y_3(t)$ ser för periodisk ut för att vara ett talat ordspråk.

$w(t)$ ser ut att existera runt 46{,}5 kHz. $y(t)$s amplitudspektrum plottas därför runt denna frekvens i figur 3. Då det är känt att $f_1$ och $f_2$ är multiplar av 1 Hz går det med figuren att dra slutsatsen att $f_1=46500$ Hz och att $f_2=46501$ Hz.

För att ta reda på tidsfördröjningen $\tau_0=\tau_2-\tau_1$ används tidsestimering med autokorrelation. Autokorrelationen definieras av $z(\tau)=\int_{-\infty}^{\infty} y(t)y(t+\tau)\mathop{dt}$. I och med att $z(\tau)$ blir större desto mer $y(t)$ liknar $y(t+\tau)$ förväntas en topp i $\tau=0$. Tidsfördröjningen förväntas vara given av det positiva $\tau$ som bortsett från toppen vid $\tau=0$ ger $z(\tau)$ sin största topp. Detta på grund av att vid detta $\tau$ är $x(t)$ i fas med sin tidsfördröjda kopia. $z(\tau)$ plottas i figur 4 och med hjälp av figuren går det att dra slutsatsen att ekots tidsfördröjning är $\tau_0=0{,}430$ ms. Detta är rimligt då det är känt att $\tau_0<500$ ms och att $\tau_0$ är en multipel av 1 ms.

Ekot filtreras bort med hjälp av ekvationen $x(\hat{t})=y(\hat{t})-0{,}9x(\hat{t}-\tau_0)$ för $\hat{t} > \tau_0$. Ekvationen härleds från ekvationen för $y(t)$ genom att sätta $\hat{t}=t+\tau_1$. Notera att $x(t)$ är den ekofria signalen och att $y(t)$ är fri från eko då $t < \tau_0$ vilket gör det möjligt att rekursivt filtrera bort ekot. Då den sökta signalen tros vara $y_1(t)$ behandlas endast den. Den ekofria varianten av $y_1(t)$ benämns hädanefter som $x_1(t)$.

Nu återstår demodulering. Ur kursbokens andra kapitel härleds uttrycken $x_{1I}(t)=\mathcal{H}^{LP}_{B/2}\{2x_1(\hat{t})\cos(2\pi f_{c}t+\delta)\}$ och $x_{1Q}(t)=\mathcal{H}^{LP}_{B/2}\{-2x_1(\hat{t})\sin(2\pi f_{c}t+\delta)\}$. Notera att $x_1(\hat{t})$ är tidsförskjuten eftersom $\hat{t}=t+\tau_1$. Tidsförskjutningen orsakar en fasvridning som i sin tur orsakar en blandning av I- och Q-komponenterna. Denna fasvridning behöver därför kompenseras för och representeras av $\delta$ i demoduleringsekvationerna. Som tidigare känt är $f_c=56$ kHz. Efter prövning sätts $\delta=1{,}1$ rad vilket verkar minimera talets överlapp i respektive komponent.

\section{Resultat}

Den sökta informationen är:
\begin{itemize}
\item Bärfrekvensen för nyttosignalen är $f_c=56$ kHz.
\item Sinusoidernas frekvenser i $w(t)$ är $f_1=46500$ Hz och $f_2=46501$ Hz.
\item Ekots bidrag till tidsfördröjningen är $\tau_2-\tau_1=430$ ms.
\item Ordspråket i $x_I(t)$ är \say{Inget ont som inte har något gott med sig}.
\item Ordspråket i $x_Q(t)$ är \say{Även den mest skröpliga mussla kan innehålla en pärla}.
\end{itemize}

\clearpage

\section*{Figurer}

\begin{figure}[H]
\includegraphics[trim={3.8cm 8cm 4.8cm 9cm},clip,width=8cm]{Y.pdf}
\caption{Amplitudspektrum för $y(t)$}
\end{figure}

\begin{figure}[H]
\includegraphics[trim={3.8cm 8cm 4.8cm 9cm},clip,width=8cm]{carried_signals.pdf}
\caption{De tre identifierade delsignalerna}
\end{figure}
\section*{ }
\begin{figure}[H]
\includegraphics[trim={3.8cm 8cm 4.8cm 9cm},clip,width=8cm]{Y_46_5kHz.pdf}
\caption{Amplitudspektrum för $y(t)$ runt $46{,}5$ kHz}
\end{figure}

\begin{figure}[H]
\includegraphics[trim={3.8cm 8cm 4.8cm 9cm},clip,width=8cm]{xcorr.pdf}
\caption{Autokorrelation av $y(t)$}
\end{figure}

\clearpage

\section*{Matlab-kod}
\small
\texttt{\lstinputlisting[language=Matlab,breaklines=true]{tsks10.m}}
\end{document}
